\documentclass{sasbase}

\usepackage{lipsum}
\usepackage{enumitem}
\usepackage{graphicx}

\begin{document}

\title{Informationsblatt der APB}
\place{Ludwigsburg}
\datum{19. November 2017}
\edition{2}

\setcounter{secnumdepth}{5}

\mytitle

% OPTIONAL
%\squarestyle
% OR
\parensstyle

\section{Politische Bildung}
Das Informationsblatt beinhaltet auch immer FAQs zu neuen Gesetzen und der Funktionsweise der staatlichen Organe.
Fragen k\"{o}nnen gerne jederzeit an den Ausschuss f\"{u}r politische Bildung gesendet werden, diese werden so fr\"{u}h als m\"{o}gliche bearbeitet und in der n\"{a}chsten Ausgabe beantwortet.

\topic{Judikative}
\begin{question}{Was sind die Aufgaben des Verfassungsgericht?}
	Das Verfassungsgericht kontrolliert, ob die vom Parlament erlassenen Gesetze verfassungskonform sind. Falls dem so nicht ist, muss das Verfassungsgericht die Erlassung der Gesetze mit einem Beschluss verhindern. Deshalb wird das Verfassungsgericht auch mit Wahl des Parlamentes zusammengestellt. Dar\"{u}ber hinaus verhandelt das Verfassungsgericht auch zivilrechtliche Klagen, die vom Straf- und Zivilgericht in Revision gegangen sind. Verfassungsrichterinnen brauchen also nicht nur ein gutes Verst\"{a}ndnis der Verfassung, sondern auch des Strafrechts.
\end{question}

\begin{question}{Was sind die Aufgaben des Zivil- und Strafgerichts?}
	Das Zivil und Strafgericht urteilt \"{u}ber alle privatrechtlichen Streitf\"{a}lle, die in Goethopia anfallen. Hierbei ist es wichtig, dass die Richterinnen gute Kenntnisse der Gesetzeslage haben und daran denken, Urteile immer zu ver\"{o}ffentlichen. Das Gericht besteht aus 3 hauptberuflichen Richterinnen und zwei Sch\"{o}ffinnen.
\end{question}

\begin{question}{Wie werde ich Sch\"{o}ffin?}
	Sch\"{o}ffinen sind jeweils nur einen Tag im Amt. Dadurch wird allen Staatsb\"{u}rgerinnen die M\"{o}glichkeit geboten, einmal Luft zu schnuppern wie das Rechtssystem funktioniert. Damit sich Arbeitgeber auf die fehlende Arbeitskraft einstellen k\"{o}nnen ist es wichtig, dass sich Sch\"{o}ffinen fr\"{u}hzeitig beim Parlament bewerben. Am 01.03. zieht das Parlament zuf\"{a}llig 10 Sch\"{o}ffinen f\"{u}r Schule als Staat.
\end{question}

\begin{question}{Wie werde ich Richterin?}
	Das Parlament w\"{a}hlt die Richterinnen selber. Als Richterin muss man keine besonderen Qualifikation mitbringen, sondern sich direkt bei der Parlamentspr\"{a}sidentin bewerben und das Parlament von der eigenen Tauglichkeit \"{u}berzeugen.
\end{question}

\topic{Posten in den Ministerien}
\begin{question}{Wie erhalte ich den Beamtenstatus?}
	Alle Ministerien (Wirtschatfs-, Innen-, Au{\ss}en-, Kultur- und Arbeitsministerium) haben Anrecht auf 3 hauptberuchliche Beamtenstellen. Sobald sich nach der Wahl eine Regierung gebildet hat, ist jede Ministerin f\"{u}r ihr Ministerium zust\"{a}ndig und kann die Stellen besetzen. Am Besten geht man auf die Minister direkt zu, wenn man sich auf eine Stelle bewerben will.
\end{question}


\section{Impressum}
Agentur f\"{u}r politische Bildung, stellvertretend Christian Merten und Nils Hebach.
\end{document}

