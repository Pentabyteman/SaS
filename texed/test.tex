\documentclass{sasbase}

\usepackage{lipsum}

\begin{document}

\title{Bundesgesetzblatt}
\place{Bonn}
\datum{23. Mai 1949}
\edition{1}

\setcounter{secnumdepth}{5}

\mytitle


\law{Grundgesetz für die Bundesrepublik Deutschland vom 23. Mai 1949}

\lipsum[1]

\segmentoflaw{Das erste Gesetz}

\squarestyle
\begin{article}[Menschenwürde]
    \label{blabla}
    \item Die Würde des Menschen ist unantastbar. blablabl
        blabla
    \item Demokratie ist toll!
\end{article}

\parensstyle
\begin{lawparagraph}[Menschenwürde]
    \item Die Würde des Menschen ist unantastbar. blablabl
        blabla
    \item Demokratie ist toll! Das ist Artikel \thearticleno
\end{lawparagraph}

\law{Weiteres Gesetz}

\begin{article}[Meinungsfreiheit]
    \item Nur richtige Meinungen sind zulässig!
\end{article}

\section{Politische Bildung}

Das sind die meist gestellten Fragen der letzten Zeit, deswegen
kommen hier die Antworten. Falls ihr mehr Fragen habt, stellt gerne
einen Antrag an den Ausschuss für politische Bildung!
Gestepped wurde in Artikel \ref{stepped}.

\topic{Parlament}

\begin{question}{Wie komme ich ins Parlament?}
    \lipsum[3]
\end{question}

\begin{question}{Wie werde ich Präsidentin?}
    \lipsum[5]
\end{question}

\begin{question}{Können auch Jungs Präsident werden?}
    Nö
\end{question}

\subsection{Blabla}

\lipsum[3]

\subsection{Noch mehr Trara!}

\lipsum[4]

\subsection{Es soll sich ja lohnen!}

\lipsum[5]

\subsection{Jetzt ist aber Schluss!}

\lipsum[6]

\subsection{Noch nicht ganz!}

\lipsum[7]

\end{document}
